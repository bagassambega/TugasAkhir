\section{Analisis Masalah}

\subsection{Analisis Sistem DecMed dan Regulasi Data Pribadi}

Sistem DecMed yang diajukan oleh I Putu Bakta Hari Sudewa \autocite{haridecmed} merupakan sistem rekam medis elektronik berbasis IOTA yang menggunakan kerangka akses kontrol.\@
Akses kontrol memungkinkan pemberian akses terhadap data kepada pihak secara spesifik dan dapat dibatasi oleh waktu yang ditentukan pasien.
Namun implementasi ini menyisakan beberapa keterbatasan, yaitu pasien harus memberikan persetujuan akses data secara langsung, dan pasien tidak dapat mengelola data apa yang akan dibagikan secara spesifik dan apa saja izin akses yang diberikan.
Hal ini tidak sesuai dengan Permenkes Nomor 24 Tahun 2022 tentang Rekam Medis Elektronik dan UU PDP yang menyatakan bahwa setiap aktor di dalam sistem rekam medis memiliki akses kontrol yang berbeda-beda, dan pemrosesan data pribadi memerlukan legalitas, jenis dan relevansi dari data pribadi yang bersangkutan.


\subsection{Identifikasi Masalah}\label{subsec:identifikasi-masalah-3}

Berdasarkan analisis sistem yang sudah dilakukan, ditemukan beberapa masalah dan keterbatasan sebagai berikut,

\begin{enumerate}
    \item Akses kontrol pada DecMed belum memberikan keleluasaan bagi pasien untuk mengelola persetujuan terhadap jenis data dan jenis izin akses data pribadinya.
    \item Pasien harus memberikan persetujuan akses data secara langsung, dan jika pasien tidak dapat mengakses gawainya, maka data pasien tidak dapat diberikan.
\end{enumerate}


\subsection{Analisis Kebutuhan}

Berdasarkan masalah yang sudah disebutkan pada subbab \@\ref{subsec:identifikasi-masalah-3}, disusun kebutuhan solusi yang perlu diimplementasikan pada tabel \@\ref{table:analisis-kebutuhan},

% Tabel kebutuhan
\begin{table}[htbp]
    \centering
    \caption{Kebutuhan Solusi untuk Sistem}
    \begin{tabular}{|p{5cm}|p{9cm}|} % chktex 44
        \hline % chktex 44
        	\textbf{Kategori} & \textbf{Deskripsi} \\
        \hline % chktex 44
        Manajemen Persetujuan Akses Data (Consent Management) & Sistem harus menyediakan portal consent yang memungkinkan pasien memberikan, mengubah, dan mencabut persetujuan akses data secara mandiri, granular (berdasarkan jenis data dan izin), dan asinkron. \\
        \hline % chktex 44
        Kontrol Akses Berbasis \textit{Authorization Framework} & Sistem harus mengimplementasikan \textit{authorization framework} untuk mengatur hak akses yang mendukung pemberian akses berdasarkan atribut (role, departemen, fasyankes, hierarki), spesifik persona, data yang diberikan dan waktu. \\
        \hline % chktex 44
        Integrasi dengan Sistem RME Terdesentralisasi (DecMed) & Portal consent dan \textit{authorization framework} harus dapat diintegrasikan dengan DecMed dan bersifat terdesentralisasi. \\
        \hline % chktex 44
        Audit dan Monitoring & Sistem harus menyediakan log aktivitas akses dan perubahan persetujuan yang dapat diaudit. \\
        \hline % chktex 44
    \end{tabular}
\end{table}\label{table:analisis-kebutuhan}



% Berdasarkan studi terhadap sistem DecMed (Sudewa, 2025) dan kondisi implementasi SATUSEHAT di Indonesia, teridentifikasi beberapa permasalahan utama terkait mekanisme kontrol akses yang perlu diselesaikan:

% \subsection{Keterbatasan Mekanisme Persetujuan Sinkron}
% Pada sistem berbasis OAuth 2.0 standar atau mekanisme \textit{smart contract} dasar seperti pada DecMed, pemberian akses seringkali bersifat sinkron. Pasien (sebagai pemilik data) harus berada dalam keadaan daring (\textit{online}) atau hadir secara fisik untuk memindai kode QR atau menyetujui permintaan akses (\textit{access request}) dari tenaga medis pada saat kejadian.

% Hal ini menimbulkan masalah ketersediaan (\textit{availability}) apabila:
% \begin{itemize}
%     \item Pasien tidak membawa gawai atau gawai kehabisan daya saat pemeriksaan.
%     \item Pasien tidak memiliki koneksi internet yang stabil.
%     \item Pasien dalam kondisi tidak dapat mengoperasikan gawai namun belum masuk kategori gawat darurat (misalnya lansia atau pasien dengan keterbatasan fisik).
% \end{itemize}
% Dalam kondisi tersebut, proses pelayanan medis dapat terhambat karena tenaga medis tidak dapat memperoleh otorisasi akses tepat waktu.

% \subsection{Granularitas dan Fleksibilitas Kebijakan Akses}
% Sistem terdahulu (DecMed) menggunakan \textit{Capability-Based Access Control} (CapBAC) dengan waktu pencabutan (\textit{revoke time}) yang cenderung statis. Setelah akses diberikan, pasien sulit untuk mengubah parameter akses secara dinamis tanpa melakukan transaksi pencabutan penuh. Selain itu, definisi kebijakan akses seringkali terbatas pada individu per individu (satu dokter spesifik).

% Kebutuhan di lapangan menuntut adanya kebijakan akses yang lebih granular dan berbasis atribut (\textit{Attribute-Based}), seperti:
% \begin{itemize}
%     \item Memberikan akses kepada seluruh tenaga medis di Poli X pada Fasyankes Y.
%     \item Memberikan akses berdasarkan hierarki, misalnya hanya kepada "Kepala Apotek".
%     \item Membatasi akses berdasarkan peran (\textit{role}) spesifik dalam durasi tertentu.
% \end{itemize}

% \subsection{Kompleksitas Manajemen Pemulihan Akun}
% Pada sistem DecMed, keamanan akun sangat bergantung pada penyimpanan mandiri 12 frasa pemulihan (\textit{seed phrase}) oleh pasien. Jika frasa ini hilang, akses terhadap rekam medis pasien akan hilang permanen karena sifat desentralisasi murni.

% Dalam konteks integrasi nasional dengan SATUSEHAT, mekanisme ini dianggap terlalu berisiko dan memiliki \textit{usability} yang rendah bagi masyarakat awam. Diperlukan mekanisme pemulihan yang lebih familiar namun tetap aman, menggunakan verifikasi identitas kependudukan (NIK, Nama Ibu Kandung, dan verifikasi biometrik/foto KTP).

% \subsection{Batasan Regulasi Akses Tulis (Write Access)}
% Regulasi medis menetapkan bahwa data klinis yang telah diinput dapat diperbarui (\textit{update}) dalam kurun waktu maksimal $2 \times 24$ jam. Sistem perlu mengakomodasi aturan logika bisnis ini di mana akses tulis (\textit{write}) tenaga medis otomatis berlaku selama periode tersebut, namun tetap menghormati hak pasien jika terjadi pencabutan akses (\textit{revocation}) secara tiba-tiba, yang kemudian memerlukan persetujuan ulang atau eskalasi ke kepala fasilitas pelayanan kesehatan (fasyankes).

% \section{Analisis Kebutuhan}

% Berdasarkan analisis masalah di atas, sistem yang dikembangkan harus memenuhi kebutuhan fungsional sebagai berikut:

% \begin{enumerate}
%     \item \textbf{Manajemen Kebijakan Asinkronus:} Sistem harus memungkinkan pasien untuk mendefinisikan kebijakan akses (\textit{policy}) kapan saja dan di mana saja tanpa harus menunggu permintaan dari tenaga medis.
%     \item \textbf{Granularitas Akses:} Sistem harus mendukung pembuatan kebijakan akses berdasarkan identitas individu, grup tenaga medis, atribut fasyankes, atau peran hierarkis.
%     \item \textbf{Integrasi SATUSEHAT:} Sistem harus menggunakan SATUSEHAT sebagai \textit{Authorization Server} (AS) terpusat untuk autentikasi dan manajemen kebijakan.
%     % \item \textbf{Mekanisme Rujukan:} Sistem harus mendukung transfer hak akses antar tenaga medis melalui mekanisme kode rujukan yang aman.
%     \item \textbf{Interoperabilitas Fasyankes:} Sistem RME di fasyankes (\textit{Resource Server}) harus dapat mendaftarkan sumber daya data pasien ke AS pusat secara otomatis.
% \end{enumerate}
