\section{Rekam Data Medis}

Menurut Peraturan Menteri Kesehatan Republik Indonesia Nomor 24 Tahun 2022 (Permenkes No. 24/2022), rekam medis adalah dokumen yang berisikan data identitas pasien, pemeriksaan, pengobatan, tindakan, dan pelayanan lain yang telah diberikan kepada pasien \parencite{permenkes2022rekam}. Lebih lanjut, Rekam Medis Elektronik (RME) didefinisikan sebagai rekam medis yang dibuat dengan menggunakan sistem elektronik yang diperuntukkan bagi penyelenggaraan rekam medis \parencite{permenkes2022rekam}. Permenkes ini mewajibkan setiap fasilitas pelayanan kesehatan (fasyankes) untuk menyelenggarakan RME paling lambat 31 Desember 2023 \parencite{permenkes2022rekam}. Fasyankes yang dimaksud mencakup tempat praktik mandiri tenaga kesehatan, puskesmas, klinik, rumah sakit, apotek, laboratorium kesehatan, balai, dan fasyankes lain yang ditetapkan Menteri \parencite{permenkes2022rekam}.

Penyelenggaraan RME bertujuan untuk meningkatkan mutu pelayanan kesehatan, memberikan kepastian hukum, menjamin keamanan, kerahasiaan, keutuhan, dan ketersediaan data, serta mewujudkan sistem RME yang berbasis digital dan terintegrasi \parencite{permenkes2022rekam}.

\subsection{Kepemilikan dan Kerahasiaan}
Permenkes No. 24/2022 menegaskan bahwa dokumen Rekam Medis adalah milik fasyankes, sedangkan isi Rekam Medis adalah milik pasien \parencite{permenkes2022rekam}. Fasyankes bertanggung jawab atas kehilangan, kerusakan, pemalsuan, atau penggunaan data oleh pihak yang tidak berhak \parencite{permenkes2022rekam}. Isi RME wajib dijaga kerahasiaannya oleh semua pihak yang terlibat, termasuk tenaga kesehatan, pimpinan fasyankes, tenaga pembiayaan, badan hukum fasyankes, mahasiswa yang bertugas, dan pihak lain yang memiliki akses \parencite{permenkes2022rekam}.

\subsection{Akses dan Keamanan}
Akses terhadap RME diatur secara ketat. Pimpinan fasyankes memberikan hak akses kepada tenaga kesehatan atau tenaga lain sesuai kebutuhan \parencite{permenkes2022rekam}. Hak akses ini meliputi penginputan data, perbaikan data (dengan batasan waktu), dan melihat data \parencite{permenkes2022rekam}. Prinsip keamanan RME mencakup kerahasiaan, integritas (keakuratan data dan perlindungan dari perubahan tidak sah), dan ketersediaan (data dapat diakses oleh pihak berwenang) \parencite{permenkes2022rekam}. Sistem elektronik yang digunakan harus memiliki kemampuan kompatibilitas dan interoperabilitas serta terhubung dengan platform yang dikelola Kementerian Kesehatan \parencite{permenkes2022rekam}.

Pentingnya keamanan dan kontrol akses dalam RME menjadi dasar perlunya mekanisme manajemen izin (consent management) yang efektif, terutama yang berpusat pada pasien, seperti yang akan dibahas lebih lanjut terkait User-Managed Access (UMA).