\section{Akses Kontrol}
Dalam konteks Rekam Medis Elektronik (RME), akses kontrol merupakan aspek yang sangat penting untuk melindungi privasi pasien dan memastikan data hanya diakses oleh pihak yang berwenang. Beberapa model akses kontrol telah dikembangkan dan diterapkan dalam sistem informasi, termasuk sistem informasi kesehatan. Studi literatur ini akan membahas beberapa model yang relevan, yaitu Role-Based Access Control (RBAC), Attribute-Based Access Control (ABAC), Capability-Based Access Control (CapBAC), OAuth 2.0, dan User-Managed Access (UMA).

\subsection{Role Based Access Control (RBAC)}
Role-Based Access Control (RBAC) adalah model akses kontrol di mana hak akses pengguna ke sumber daya sistem ditentukan oleh peran (\textit{role}) yang terkait dengan pengguna tersebut \autocite{ferraiolo1992role}. Dalam model ini, izin \textit{(permission)} tidak diberikan langsung kepada pengguna, tetapi dikaitkan dengan peran. Pengguna kemudian ditugaskan ke peran yang sesuai dengan fungsi atau jabatan mereka dalam organisasi \autocite{carvalho2018health}.

RBAC banyak digunakan dalam sistem informasi kesehatan yang sudah ada karena kemampuannya menyederhanakan administrasi hak akses dalam organisasi yang kompleks \autocite{carvalho2018health}. Dengan mengelola peran daripada pengguna individu, administrator dapat lebih mudah mengatur siapa yang dapat mengakses apa. Namun, RBAC memiliki beberapa tantangan dalam konteks RME. Salah satunya adalah \textit{role explosion}, di mana jumlah peran bisa menjadi sangat banyak dan sulit dikelola seiring dengan meningkatnya kompleksitas kebutuhan akses \autocite{carvalho2018health}. Selain itu, RBAC bersifat statis dan kurang fleksibel dalam mengakomodasi kebutuhan akses yang dinamis atau berbasis konteks, seperti kondisi darurat atau kebutuhan akses pasien itu sendiri terhadap data mereka \autocite{carvalho2018health, alhaqbani2008access}.

\subsection{Attribute Based Access Control (ABAC)}
Attribute-Based Access Control (ABAC) adalah model akses kontrol di mana keputusan akses didasarkan pada atribut yang terkait dengan pengguna (subjek), sumber daya (objek), lingkungan (environment), dan kebijakan (policy) \autocite{hu2013attribute}. Kebijakan ini mendefinisikan aturan yang menentukan apakah akses diizinkan atau ditolak berdasarkan evaluasi atribut-atribut tersebut pada saat permintaan akses \autocite{yaqub2025blockchain}.

Contoh atribut pengguna bisa berupa peran, departemen, atau kualifikasi. Atribut sumber daya bisa berupa jenis data (misalnya, rekam medis, hasil lab), tingkat sensitivitas, atau pemilik data. Atribut lingkungan bisa mencakup waktu akses, lokasi, atau status sistem.

ABAC menawarkan fleksibilitas yang lebih besar dibandingkan RBAC karena keputusan akses dapat dibuat secara dinamis berdasarkan konteks yang lebih kaya \autocite{yaqub2025blockchain}. Hal ini membuatnya lebih cocok untuk lingkungan yang kompleks dan dinamis seperti pelayanan kesehatan, di mana kebutuhan akses seringkali bergantung pada situasi spesifik pasien atau kondisi operasional \autocite{imam2024practically}. Penerapan ABAC dengan teknologi blockchain juga telah diusulkan untuk meningkatkan keamanan dan privasi RME \autocite{yaqub2025blockchain}. Namun, implementasi ABAC bisa menjadi kompleks karena memerlukan definisi kebijakan yang rinci dan manajemen atribut yang cermat.

\subsection{Capability Based Access Control (CapBAC)}
Capability-Based Access Control (CapBAC) adalah model akses kontrol di mana hak akses direpresentasikan oleh token yang tidak dapat dipalsukan yang disebut kapabilitas \autocite{sandhu1996capability}. Kapabilitas ini menggabungkan identitas objek dan hak akses yang diizinkan untuk objek tersebut. Pengguna (atau proses yang bertindak atas nama pengguna) yang memiliki kapabilitas dapat menyajikannya ke sistem untuk mendapatkan akses ke sumber daya yang sesuai. Kepemilikan kapabilitas itu sendiri merupakan bukti otorisasi \autocite{sandhu1996capability}.

Dalam konteks RME, kapabilitas dapat digunakan untuk memberikan akses granular kepada dokter atau pasien ke bagian tertentu dari rekam medis untuk jangka waktu tertentu. Keuntungan CapBAC adalah potensi desentralisasi dalam manajemen akses dan kemudahan delegasi (pengguna dapat memberikan kapabilitasnya kepada pengguna lain). Namun, manajemen kapabilitas, terutama pencabutan (revocation), bisa menjadi tantangan dalam sistem terdistribusi skala besar.

\subsection{OAuth 2.0}
OAuth 2.0 bukanlah model akses kontrol, melainkan sebuah kerangka kerja otorisasi (authorization framework) \autocite{hardt2012oauth}. OAuth 2.0 memungkinkan aplikasi pihak ketiga (client) untuk mendapatkan akses terbatas ke sumber daya HTTP atas nama pemilik sumber daya (resource owner), tanpa mengungkapkan kredensial pemilik sumber daya ke aplikasi pihak ketiga tersebut \autocite{hardt2012oauth}.

OAuth 2.0 mendefinisikan empat peran utama:
\begin{enumerate}
    \item \textbf{Resource Owner}: Entitas yang dapat memberikan akses ke sumber daya yang dilindungi (misalnya pasien).
    \item \textbf{Resource Server}: Server yang menyimpan sumber daya yang dilindungi (misalnya server RME fasyankes).
    \item \textbf{Client}: Aplikasi yang meminta akses ke sumber daya atas nama Resource Owner (misalnya aplikasi portal pasien atau aplikasi dokter).
    \item \textbf{Authorization Server}: Server yang mengotentikasi Resource Owner dan mengeluarkan token akses kepada Client setelah mendapatkan persetujuan dari Resource Owner.
\end{enumerate}
Dalam konteks RME, OAuth 2.0 dapat digunakan untuk memungkinkan pasien (Resource Owner) memberikan izin kepada aplikasi dokter (Client) untuk mengakses RME mereka yang disimpan di server fasyankes (Resource Server), melalui mekanisme persetujuan yang dikelola oleh Authorization Server \autocite{imam2024practically}. OAuth 2.0 menyediakan dasar untuk membangun sistem persetujuan yang terdelegasi, tetapi tidak secara spesifik mendefinisikan bagaimana persetujuan itu dikelola oleh pemilik sumber daya.

\subsection{User Managed Access (UMA)}
User-Managed Access (UMA) 2.0 adalah profil standar dari OAuth 2.0 yang dirancang khusus untuk memungkinkan pemilik sumber daya (Resource Owner) mengelola otorisasi akses ke sumber daya mereka secara terpusat dan dinamis \autocite{kantarcioglu2007uma}. UMA memperluas OAuth 2.0 dengan menambahkan peran dan alur kerja baru untuk memfasilitasi manajemen persetujuan yang berpusat pada pengguna (owner-managed).

Fitur utama UMA meliputi:
\begin{enumerate}
    \item \textbf{Manajemen Kebijakan Terpusat}: Resource Owner dapat menetapkan kebijakan akses (siapa dapat mengakses apa, kapan, dan dalam kondisi apa) pada Authorization Server (AS) terpusat.
    \item \textbf{Persetujuan Asinkron}: Pihak yang meminta akses (Requesting Party) tidak perlu berinteraksi langsung dengan Resource Owner. AS bertindak sebagai perantara, mengevaluasi permintaan akses terhadap kebijakan yang ditetapkan oleh Resource Owner, dan jika perlu, memfasilitasi proses persetujuan (consent).
    \item \textbf{Akses Terdelegasi}: Resource Owner dapat mendelegasikan hak pengelolaan akses kepada pihak lain.
\end{enumerate}
Dalam konteks portal persetujuan pasien berbasis RME, UMA menyediakan kerangka kerja standar yang sangat fleksibel. Pasien (Resource Owner) dapat menggunakan portal (yang berinteraksi dengan AS) untuk mengatur siapa saja (dokter, keluarga, aplikasi lain) yang boleh mengakses bagian mana dari RME mereka, dan dalam kondisi apa. Ketika seorang dokter (Requesting Party) mencoba mengakses RME pasien melalui sistem fasyankes (Resource Server), sistem akan mengarahkan dokter ke AS. AS akan mengevaluasi permintaan berdasarkan kebijakan pasien. Jika kebijakan memerlukan persetujuan eksplisit, AS dapat memberitahu pasien (misalnya melalui portal atau notifikasi) untuk menyetujui atau menolak permintaan tersebut. Jika disetujui, AS akan mengeluarkan token akses (berbasis OAuth 2.0) kepada dokter.

UMA secara langsung mendukung konsep \textit{owner-managed consent}, sehingga memberikan kontrol penuh kepada pasien atas data RME mereka, yang sejalan dengan prinsip kepemilikan isi RME oleh pasien sebagaimana diatur dalam Permenkes No. 24/2022 \autocite{permenkes2022rekam}.

% ga usah bandingin dengan akses kontrol        lain