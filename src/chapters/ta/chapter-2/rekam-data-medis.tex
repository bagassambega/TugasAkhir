\section{Rekam Medis Elektronik}

Menurut Peraturan Menteri Kesehatan Republik Indonesia Nomor 24 Tahun 2022, rekam medis adalah dokumen yang berisikan data identitas pasien, pemeriksaan, pengobatan, tindakan, dan pelayanan lain yang telah diberikan kepada pasien. 
Lebih lanjut, rekam medis elektronik (RME) didefinisikan sebagai rekam medis yang dibuat dengan menggunakan sistem elektronik yang diperuntukkan bagi penyelenggaraan rekam medis. 


% Permenkes ini mewajibkan setiap fasilitas pelayanan kesehatan (fasyankes) untuk menyelenggarakan RME-nya masing-masing. 
% Fasyankes yang dimaksud mencakup tempat praktik mandiri tenaga kesehatan, puskesmas, klinik, rumah sakit, apotek, laboratorium kesehatan, balai, dan fasyankes lain yang ditetapkan Menteri \autocite{permenkes2022rekam}.

% Penyelenggaraan RME bertujuan untuk meningkatkan mutu pelayanan kesehatan, memberikan kepastian hukum, menjamin keamanan, kerahasiaan, keutuhan, dan ketersediaan data, serta mewujudkan sistem RME yang berbasis digital dan terintegrasi \autocite{permenkes2022rekam}.

\subsection{Isi dan Aktor Rekam Medis}

Permenkes No. 24 Tahun 2022 menegaskan bahwa dokumen Rekam Medis adalah milik fasyankes, sedangkan isi Rekam Medis adalah milik pasien. 
Fasyankes bertanggung jawab atas kehilangan, kerusakan, pemalsuan, atau penggunaan data oleh pihak yang tidak berhak. 
Isi RME wajib dijaga kerahasiaannya oleh semua pihak yang terlibat, termasuk tenaga kesehatan, pimpinan fasyankes, tenaga pembiayaan, badan hukum fasyankes, mahasiswa yang bertugas, dan pihak lain yang memiliki akses \autocite{permenkes2022rekam}.


\subsection{Fast Healthcare Interoperability Resources (FHIR)}



% \subsection{Akses dan Keamanan}
% Akses terhadap RME diatur secara ketat. Pimpinan fasyankes memberikan hak akses kepada tenaga kesehatan atau tenaga lain sesuai kebutuhan \autocite{permenkes2022rekam}. Hak akses ini meliputi penginputan data, perbaikan data (dengan batasan waktu), dan melihat data \autocite{permenkes2022rekam}. Prinsip keamanan RME mencakup kerahasiaan, integritas (keakuratan data dan perlindungan dari perubahan tidak sah), dan ketersediaan (data dapat diakses oleh pihak berwenang) \autocite{permenkes2022rekam}. Sistem elektronik yang digunakan harus memiliki kemampuan kompatibilitas dan interoperabilitas serta terhubung dengan platform yang dikelola Kementerian Kesehatan \autocite{permenkes2022rekam}.

% \subsection{Kepemilikan dan Kerahasiaan}

% Pentingnya keamanan dan akses kontrol dalam RME menjadi dasar perlunya mekanisme manajemen izin (consent management) yang efektif, terutama yang berpusat pada pasien, seperti yang akan dibahas lebih lanjut terkait User-Managed Access (UMA).

% \subsection{Akses dan Keamanan}
% Akses terhadap RME diatur secara ketat. Pimpinan fasyankes memberikan hak akses kepada tenaga kesehatan atau tenaga lain sesuai kebutuhan \autocite{permenkes2022rekam}. Hak akses ini meliputi penginputan data, perbaikan data (dengan batasan waktu), dan melihat data \autocite{permenkes2022rekam}. Prinsip keamanan RME mencakup kerahasiaan, integritas (keakuratan data dan perlindungan dari perubahan tidak sah), dan ketersediaan (data dapat diakses oleh pihak berwenang) \autocite{permenkes2022rekam}. Sistem elektronik yang digunakan harus memiliki kemampuan kompatibilitas dan interoperabilitas serta terhubung dengan platform yang dikelola Kementerian Kesehatan \autocite{permenkes2022rekam}.
