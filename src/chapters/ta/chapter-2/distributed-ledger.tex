\section{Distributed Ledger Technology (DLT)}

Distributed Ledger Technology (DLT), yang paling dikenal melalui implementasinya dalam bentuk Blockchain, adalah sistem pencatatan data yang terdistribusi dan tersinkronisasi di antara banyak partisipan dalam jaringan \autocite{liu2017medical, yaqub2025blockchain}. Karakteristik utama DLT meliputi:
\begin{itemize}
    \item Desentralisasi: Tidak ada otoritas pusat tunggal yang mengendalikan ledger. Data disimpan dan divalidasi oleh banyak node dalam jaringan.
    \item Immutability (Kekekalan): Setelah data dicatat dalam ledger (dalam sebuah blok yang ditambahkan ke rantai/chain), sangat sulit atau hampir tidak mungkin untuk mengubah atau menghapusnya tanpa persetujuan mayoritas jaringan. Ini dicapai melalui penggunaan fungsi hash kriptografis yang menghubungkan setiap blok dengan blok sebelumnya \autocite{liu2017medical}.
    \item Transparansi (dengan Pseudonimitas): Transaksi yang tercatat dalam ledger seringkali dapat dilihat oleh semua partisipan (tergantung pada jenis blockchain: publik atau privat/konsorsium), meskipun identitas asli partisipan dapat disamarkan melalui penggunaan alamat pseudonim.
    \item Keamanan Kriptografis: Transaksi diamankan menggunakan teknik kriptografi seperti digital signature untuk memastikan otentikasi dan integritas \autocite{liu2017medical, yaqub2025blockchain}.
\end{itemize}

Dalam konteks Rekam Medis Elektronik (RME), DLT memberikan peningkatan keamanan, integritas, transparansi, dan auditabilitas data \autocite{liu2017medical, yaqub2025blockchain}. Penggunaannya dapat membantu mengatasi beberapa tantangan dalam pengelolaan RME, terutama terkait akses kontrol dan privasi.

Penggunaan DLT dalam RME, seperti yang diusulkan oleh Yaqub et al. \autocite{yaqub2025blockchain}, dapat memperkuat mekanisme akses kontrol berbasis kebijakan (seperti ABAC) dengan menyediakan lapisan auditabilitas dan integritas tambahan. Hal ini sejalan dengan kebutuhan akan keamanan dan auditabilitas yang tinggi dalam pengelolaan data kesehatan sensitif.