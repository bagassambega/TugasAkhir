\section{Jadwal}

Jadwal pengerjaan Tugas Akhir ini dirancang untuk periode Februari hingga Juli 2026, dengan mengikuti tahapan metodologi yang telah ditetapkan. Gambar \ref{fig:jadwal-pengerjaan} menunjukkan timeline pengerjaan secara rinci dalam bentuk diagram Gantt.

\begin{figure}[htbp]
    \centering
    \includegraphics[width=\textwidth]{resources/chapter-4/jadwal.png}
    \caption{Jadwal Pengerjaan Tugas Akhir}\label{fig:jadwal-pengerjaan}
\end{figure}

Jadwal pengerjaan dibagi menjadi empat tahap utama sebagai berikut:

\begin{enumerate}
    \item \textbf{Studi Literatur dan Analisis Sistem Saat Ini (Februari--Maret 2026):} Tahap ini melakukan analisis sistem DecMed dan studi literatur mengenai authorization framework untuk memahami arsitektur dan mekanisme yang akan diintegrasikan.
    
    \item \textbf{Perancangan Arsitektur dan Desain Sistem (Maret--April 2026):} Tahap ini menyusun spesifikasi kebutuhan, mendesain arsitektur sistem, dan membuat model alur otorisasi serta consent management.
    
    \item \textbf{Pengembangan Prototipe Sistem (April--Juni 2026):} Tahap ini mengimplementasikan prototipe, mengintegrasikan komponen authorization framework dan portal consent dengan DecMed, serta menyusun skenario pengujian.
    
    \item \textbf{Pengujian dan Evaluasi (Juni--Juli 2026):} Tahap ini melakukan pengujian fungsionalitas sistem dan menganalisis hasil evaluasi untuk menilai pencapaian tujuan yang telah ditetapkan.
\end{enumerate}

