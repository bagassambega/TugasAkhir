\section{Rumusan Masalah}

Berdasarkan latar belakang yang sudah dijelaskan pada bab I.1, peneliltian ini berfokus pada implementasi \textit{consent portal} pasien berbasis kerangka otorisasi UMA pada sistem rekam medis elektronik berbasis DLT.\@
Berikut adalah rumusan masalah yang akan dijawab dalam penelitian ini:

\begin{enumerate}
    \item Bagaimana rancangan \textit{consent portal} pasien berbasis kerangka otorisasi UMA dapat diimplemantasikan dalam sistem rekam medis elektronik berbasis DLT?\@
    \item Bagaimana implementasi \textit{consent portal} pasien berbasis kerangka otorisasi UMA dapat meningkatkan keamanan pada segi kerahasiaan dan integritas data dalam sistem rekam medis elektronik berbasis DLT?\@
    \item Bagaimana perbandingan kerangka otorisasi UMA terhadap mekanisme akses kontrol lainnya dilihat dari sisi \textit{usability}, \textit{SLA approval}, \textit{auditability}, dan \textit{revoke time}?
\end{enumerate}

% ganti integritas data dengan privasi, tapi ini harus dijelasin dan dibandingin dengan state awal

% perbandingan auth framework bukan access control

% Implementasi Authorization Framework pada .... berbasis DLT

% tambahin related work