\section{Latar Belakang}

\sloppy

Kesehatan merupakan salah satu aspek fundamental dalam hak asasi manusia 
\parencite{standarnormanomor4kesehatankomnasham}. Hal ini dijamin oleh negara melalui penyelenggaraan kesehatan dan diatur melalui perundang-undangan yang menyatakan 
bahwa negara menjamin hak setiap warga negara untuk mewujudkan kehidupan yang baik, sehat, serta sejahtera lahir dan batin \parencite{uunomor17tahun2023}.

Penyelenggaraan pelayanan kesehatan di suatu wilayah atau suatu negara memerlukan mekanisme pencatatan pelayanan kesehatan dan informasi medis pasien yang disimpan dalam bentuk rekam medis. 
Penyelenggaraan pelayanan kesehatan yang dinamis dan membutuhkan kolaborasi antar-fasilitas pelayanan kesehatan mendorong terciptanya sistem rekam medis berbasis digital atau biasa disebut Rekam Medis Elektronik (RME).
Untuk memenuhi kebutuhan tersebut, Pemerintah Republik Indonesia sudah membangun sistem rekam medis elektronik terintegrasi yang dinamakan SatuSehat, dan diatur melalui Peraturan Menteri Kesehatan Nomor 24 Tahun 2022.

Sistem rekam medis elektronik memerlukan sistem keamanan dan akses kontrol yang komprehensif karena berkaitan dengan data dan privasi pasien. 
Berdasarkan Peraturan Menteri Kesehatan Nomor 24 Tahun 2022, isi rekam medis adalah milik pasien dan harus dijaga kerahasiaannya \parencite{permenkes2022rekam}. 
Berbagai penelitian dan pengembangan sudah dilakukan untuk melindungi kerahasiaan dan integritas sistem rekam medis elektronik, salah satunya dengan teknologi \textit{Distributed Ledger Technology} (DLT).

Salah satu sistem rekam medis elektronik yang sudah mengimplementasikan teknologi DLT dalam sistemnya adalah MedRec yang dikembangkan oleh MIT.\@
Sistem ini mengimplementasikan teknologi \textit{blockchain} dan \textit{capability based access control} (CapBAC) yang disimpan sebagai smart contract untuk mengatur pihak-pihak yang dapat mengakses rekam medis \parencite{medrec}. 
Namun implementasi \textit{blockchain} pada sistem rekam medis elektronik memberikan tantangan baru berupa tambahan \textit{latency} saat mengakses data karena menggunakan konsensus 
\textit{Proof-of-Work}, sehingga rata-rata dibutuhkan waktu hingga 10 detik untuk menetapkan menetapkan kontrak baru dan merespons.

Solusi lainnya yang diberikan adalah DecMed, implementasi sistem rekam medis elektronik berbasis IOTA yang dikembangkan oleh I Putu Bakta Hari Sudewa, mahasiswa S1 Teknik Informatika 
Institut Teknologi Bandung angkatan 2021. Penggunaan IOTA memberikan penyelesaian permasalahan sistem rekam medis elektronik berbasis \textit{blockcchain} yang memiliki komputasi mahal.
Implementasi DecMed sendiri menggunakan mekanisme akses kontrol \textit{capability based access control} (CapBAC) yang dikombinasikan dengan \textit{proxy re-encryption} dan  \textit{revoke time}.

Implementasi DecMed juga masih menyisakan kelemahan dalam sistemnya, di antaranya,
\begin{itemize}
    \item Pasien harus melakukan \textit{scan} suatu QR code yang diberikan oleh fasyankes untuk memberikan izin akses terhadap datanya di fasyankes. 
    Hal ini membuat pasien harus hadir secara langsung untuk memberikan akses.
    % \item Sistem ini hanya mendefinisikan akses kontrol untuk aktor yang telah ditentukan sebelumnya (tenaga medis, tenaga administrasi, dll). 
    % Menambahkan jenis pihak pemohon baru (misalnya, perusahaan asuransi, lembaga penelitian, layanan darurat) memerlukan modifikasi \textit{smart contract} dan aplikasi klien.
    \item Tidak ada cara bagi klien untuk mengetahui sumber daya apa yang dilindungi atau izin apa yang dibutuhkan. 
    Klien hanya dapat memberikan akses untuk data administratif dan data medisnya saja.
    
\end{itemize}
Untuk mengatasi ketiga masalah ini, diusulkan penggunaan User Managed Access (UMA) sebagai mekanisme otorisasi dan juga implementasi \textit{consent portal} bagi pasien.
UMA memberikan kemampuan kepada pasien untuk memberikan izin secara asinkronus, mengelola data mana yang akan diatur aksesnya dan mengatur pihak yang akan mengakses data miliknya dan seluruh kapabilitasnya.
