\section{Latar Belakang}\label{subsec:latar-belakang}

\sloppy

Penyelenggaraan tata kelola klinis di Indonesia saat ini mewajibkan setiap fasilitas pelayanan kesehatan (Fasyankes) untuk menyelenggarakan rekam medis elektronik (RME) secara mandiri sebagai basis dokumentasi medis pasien \autocite{permenkes2022rekammedis}.
% Kewajiban ini mengharuskan seluruh fasyankes, mulai dari tingkat pratama hingga tingkat lanjut, untuk melakukan transisi penuh dari rekam medis berbasis kertas menjadi sistem elektronik.
Dengan demikian, arsitektur data kesehatan nasional bertumpu pada ribuan sistem RME yang dikelola oleh masing-masing entitas penyedia layanan kesehatan \autocite{kemkesBerita}.
Agar data yang tersebar di berbagai sistem RME tersebut dapat dipertukarkan untuk kesinambungan pelayanan, pemerintah menetapkan bahwa setiap sistem elektronik milik fasyankes harus memiliki kemampuan interoperabilitas melalui platform SATUSEHAT \autocite{permenkes2022satusehat}.
% SATUSEHAT merupakan ekosistem pertukaran data kesehatan yang menghubungkan sistem informasi atau aplikasi dari seluruh anggota ekosistem digital kesehatan Indonesia termasuk fasyankes, regulator, penjamin, dan penyedia layanan digital \autocite{kemkesSATUSEHATEkosistem}.

Secara arsitektur, SATUSEHAT didesain sebagai \textit{National Health Data Platform} yang berfungsi sebagai hub integrasi tunggal dan integrasi secara terpusat. 
% Meskipun penyimpanan fisik data klinis (RME) tetap berada di masing-masing fasyankes, namun mekanisme pertukaran data, indeks pasien, dan manajemen persetujuan (\textit{consent management}) dikelola secara terpusat oleh Kementerian Kesehatan \autocite{cetakbirukemenkes}.
% Arsitektur tersentralisasi pada sistem rekam medis memberikan kemudahan dalam mekanisme interoperabilitas data, namun memberikan tantangan baru berupa \textit{single point of failure} \autocite{kumar2025decentralizedAIhealth}. 
Masalah yang dihadapi oleh sistem rekam medis terpusat suatu modul atau komponen sistem mengalami kegagalan, maka seluruh sistem dapat mengalami kegagalan total \autocite{jin2020blockchainIPFSmedicalrecord}.
Selain itu, sistem yang tersentralisasi juga menjadikan sistem rekam medis target utama serangan siber yang dapat melumpuhkan operasional rumah sakit dan membahayakan keselamatan pasien \autocite{kumar2025decentralizedAIhealth}.

Berbagai implementasi sistem rekam medis terdesentralisasi seperti blockchain, Ethereum, dan IOTA sudah digunakan untuk menciptakan sistem rekam medis yang terdesentralisasi \autocite{zhengdltcomparisonmedicalrecord}.
Dari berbagai teknologi desentralisasi tersebut IOTA merupakan teknologi desentralisasi yang paling efisien karena tidak membutuhkan biaya (\textit{gas fee}) dan \textit{resources} yang tinggi dan juga terdesentralisasi penuh \autocite{zhengdltcomparisonmedicalrecord}.

Berdasarkan Permenkes Nomor 24 Tahun 2022 juga, sistem rekam medis elektronik berperan dalam mengelola data administratif dan data klinis pasien \autocite{permenkes2022rekammedis}. 
Kedua data ini termasuk ke dalam data pribadi dan harus dijaga kerahasiaannya \autocite{uudatapribadi}.
Oleh karena itu selain solusi desentralisasi, diperlukan juga suatu mekanisme pemeriksaan otorisasi untuk memastikan akses terhadap data pasien bisa dikelola dengan baik, yaitu dengan implementasi \textit{framework} otorisasi (\textit{authorization framework}) \autocite{kheng2022delegatedauthhealthrecord}.
\textit{Authorization framework} adalah kerangka keamanan umum yang mencakup autentikasi dan otorisasi pengguna \autocite{ammarauthorizationframework}.

Selain pemeriksaan otorisasi, penggunaan dan pemrosesan data pribadi pasien juga memerlukan izin dan persetujuan dari pasien itu sendiri karena setiap pasien berhak atas kerahasiaan data pribadinya \autocite{uudatapribadi}.
Persetujuan penggunaan data pasien dapat dikelola melalui portal perizinan (\textit{consent portal}) yang berisi mekanisme pengelolaan persetujuan secara konkrit dan terinformasi (\textit{informed consent}) \autocite{dara2020patientconsent}.

\textit{Consent portal} merupakan aspek esensial dalam suatu sistem karena sistem harus memberikan kontrol kepada pemilik data yang merupakan bagian dari privasi mereka dalam rangka pemenuhan regulasi perlindungan data pribadi \autocite{prasanthconsentmanagement}.
% Persetujuan yang diberikan oleh pasien harus memberikan keleluasaan bagi pasien untuk mengatur kepada siapa data diberikan, data apa saja yang diberikan, dan bagaimana pasien mengelola pemberian/pencabutan akses terhadap data tersebut \autocite{kellygranularaccessdata,dara2020patientconsent}.
Consent portal berperan sebagai mekanisme pendefinisian regulasi mengenai data, pihak dan akses apa saja yang bisa diberikan kepada pihak tertentu \autocite{kellygranularaccessdata,dara2020patientconsent}.
% Lebih lanjut lagi, keleluasaan waktu pemberian persetujuan juga harus dipertimbangkan untuk mengatasi kasus saat pasien tidak dapat memberikan persetujuan \autocite{kheng2022delegatedauthhealthrecord}.

% Penyelenggaraan pelayanan kesehatan di suatu negara memerlukan mekanisme pencatatan pelayanan kesehatan dan informasi medis pasien yang disimpan dalam bentuk rekam medis. 
% Penyelenggaraan pelayanan kesehatan yang dinamis dan membutuhkan kolaborasi antar-fasilitas pelayanan kesehatan mendorong terciptanya sistem rekam medis berbasis digital atau biasa disebut Rekam Medis Elektronik (RME).
% Untuk memenuhi kebutuhan tersebut, Pemerintah Republik Indonesia sudah membangun sistem rekam medis elektronik terintegrasi yang dinamakan SatuSehat, dan diatur melalui Peraturan Menteri Kesehatan Nomor 24 Tahun 2022.

% Sistem rekam medis elektronik memerlukan sistem keamanan dan akses kontrol yang komprehensif karena berkaitan dengan data dan privasi pasien. 
% Berdasarkan Peraturan Menteri Kesehatan Nomor 24 Tahun 2022, isi rekam medis adalah milik pasien dan harus dijaga kerahasiaannya \autocite{permenkes2022rekammedis}. 
% Berbagai penelitian dan pengembangan sudah dilakukan untuk melindungi kerahasiaan dan integritas sistem rekam medis elektronik, salah satunya dengan teknologi \textit{Distributed Ledger Technology} (DLT).

% Salah satu sistem rekam medis elektronik yang sudah mengimplementasikan teknologi DLT dalam sistemnya adalah MedRec yang dikembangkan oleh MIT.
% Sistem ini mengimplementasikan teknologi \textit{blockchain} dan \textit{capability based access control} (CapBAC) yang disimpan sebagai smart contract untuk mengatur pihak-pihak yang dapat mengakses rekam medis \autocite{medrec}. 
% Namun implementasi \textit{blockchain} pada sistem rekam medis elektronik memberikan tantangan baru berupa tambahan \textit{latency} saat mengakses data karena menggunakan konsensus 
% \textit{Proof-of-Work}, sehingga rata-rata dibutuhkan waktu hingga 10 detik untuk menetapkan menetapkan kontrak baru dan merespons.


Salah satu solusi untuk mengatasi permasalahan rekam medis tersentralisasi dan memberikan kerahasiaan data pasien adalah DecMed, implementasi sistem rekam medis elektronik berbasis IOTA yang dikembangkan oleh I Putu Bakta Hari Sudewa, mahasiswa S1 Teknik Informatika Institut Teknologi Bandung angkatan 2021. 
Penggunaan IOTA memberikan penyelesaian permasalahan sistem rekam medis elektronik yang tersentralisasi namun tetap menggunakan sistem dengan operasi komputasi yang tidak mahal.
Implementasi DecMed sendiri menggunakan mekanisme akses kontrol \textit{capability-based access control} untuk menjamin hanya pihak tertentu saja yang bisa mengakses data pasien, dan memungkinkan pasien memberikan akses kepada pihak secara spesifik dan bisa dicabut kapan saja.

Namun, implementasi DecMed juga masih menyisakan kelemahan dalam sistemnya, di antaranya,
\begin{itemize}
    \item Pasien harus memberikan persetujuan akses data secara langsung (\textit{synchronous}) dengan melakukan \textit{scan} suatu QR code yang diberikan oleh fasyankes untuk memberikan izin akses terhadap datanya. 
    Hal ini membuat pasien harus hadir secara langsung untuk memberikan akses dan pasien tidak dapat memberikan izin kepada pihak lain saat pasien tidak sadar atau tidak bisa untuk menggunakan perangkatnya.
    % \item Sistem ini hanya mendefinisikan akses kontrol untuk aktor yang telah ditentukan sebelumnya (tenaga medis, tenaga administrasi, dll). 
    % Menambahkan jenis pihak pemohon baru (misalnya, perusahaan asuransi, lembaga penelitian, layanan darurat) memerlukan modifikasi \textit{smart contract} dan aplikasi klien.
    \item Tidak ada cara bagi klien untuk menentukan sumber daya apa yang akan diberikan dan izin apa saja yang diberikan kepada pihak lain. 
    Pasien hanya memberikan persetujuan kepada pihak lain secara spesifik, namun tidak bisa mengelola izin pakai apa yang bisa dilakukan oleh pihak tersebut, seperti \textit{read}, \textit{write}, dan kapabilitas lainnya.
    
\end{itemize}

Berdasarakan dua permasalahan ini, Tugas Akhir mengusulkan penggunaan \textit{authorization framework} dan \textit{consent portal} bagi pasien dalam sistem rekam medis DecMed.
\textit{Authorization framework} diimplementasikan untuk mengatasi akses kontrol yang tidak memadai untuk kebutuhan kompleks seperti otorisasi secara asinkronus, delegasi dan pemberian akses granular yang diperlukan dalam sistem terdistribusi modern \autocite{jaimeauthorizationmodels}.
Selain itu, implementasi \textit{consent portal} bagi pasien juga dilakukan untuk memberikan kontrol yang lebih granular dan keleluasaan dalam penggunaannya \autocite{dara2020patientconsent}.


% UMA memberikan kemampuan kepada pasien untuk memberikan izin secara asinkronus, mengelola data mana yang akan diatur aksesnya dan mengatur pihak yang akan mengakses data miliknya dan seluruh kapabilitasnya.

% jelasin consent portal
% hapus hukum yang general, lebih jelaskan yang terkait kepemilikan data di Indonesia punya rekam medis nya masing2 di fasyankes
%jadiiin paragraf untuk kelemahan
%mapping masalah ke latar belakang
% di rumusan masalah tambahkan untuk apa
% tambahin flow alur ke UMA mengapa bisa menyelesaikan Decmed  (bridging)
% atau di bab 3 jelasin mapping UMA ke decmed, di bab awal. jelasin UMA bagusnya buat apa, lalu       mapping kenapa UMA bisa menyelesaikan masalah di DecMed


% Kondisi rekam medis di Indonesia (SATUSEHAT)
% yang saat ini sudah mau menyelesaikan baru Kak Hari, tapi yang kak hari masih kurang dari segi ...
% nanti diselesaikan dengan UMA

% ganti judul dengan DLT
% cari paper yang sudah bisa menyelesaikan masalah ... dengan UMA, nanti dikaitkan dengan yang Kak Hari
% emangnya bisa langsung implementasi UMA? jadi jelaskan juga tantangan dan analisis apa aja yang perlu dilakuin sebelum implementasi, ga ujug2 cuman implementasi
% jelasin penyesuaian framework UMA ke sistem ini juga

% pakai decentralized untuk autorisasi dan jelasin tradeoff nya. tapi menjaga degree tetep biar ga terpusat

% semua istilah kaya SLA approval, auditability, dsb sebutin briefly di pendahuluan

% tujuan dan ukuran keberhasilan harus sejumlah dengan rumusan masalah
% tujuan utama dan khusus
% kalau mau nambahin yang pendaftaran fasyankes ke sistem terpusat, maka masukin juga di bab 1 bab 2

% jelasin kasus spesifiknya atau di bagian awal jelasinnya mekanisme otorisasi
