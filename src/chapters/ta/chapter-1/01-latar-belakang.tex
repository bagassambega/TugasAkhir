\section{Latar Belakang}

Kesehatan merupakan salah satu aspek fundamental dan termasuk dalam hak asasi yang harus dimiliki oleh manusia 
\parencite{standarnormanomor4kesehatankomnasham}. Hal in dijamin oleh negara melalui penyelenggaraan kesehatan dan diatur melalui perundang-undangan yang menyatakan 
bahwa negara menjamin hak setiap warga negara untuk mewujudkan kehidupan yang baik, sehat, serta sejahtera lahir dan batin \parencite{uunomor17tahun2023}.

Penyelenggaraan pelayanan kesehatan di suatu wilayah atau suatu negara memerlukan mekanisme pencatatan pelayanan kesehatan dan informasi medis pasien yang disebut sebagai rekam medis. 
Penyelenggaraan pelayanan kesehatan yang dinamis dan membutuhkan kolaborasi antar-fasilitas pelayanan kesehatan mendorong terciptanya sistem 
rekam medis berbasis digital atau biasa disebut Rekam Medis Elektronik (RME).
Untuk memenuhi kebutuhan tersebut, Pemerintah Republik Indonesia sudah membangun sistem rekam medis elektronik terintegrasi yang dinamakan SatuSehat, dan diatur melalui Peraturan 
Menteri Kesehatan Nomor 24 Tahun 2022.

Sistem rekam medis elektronik memerlukan sistem keamanan dan akses kontrol yang komprehensif karena berkaitan dengan data dan privasi pasien. Berdasarkan Peraturan 
Menteri Kesehatan Nomor 24 Tahun 2022, isi rekam medis adalah milik pasien dan harus dijaga kerahasiaannya \parencite{permenkes2022rekam}. Berbagai penelitian dan pengembangan 
sudah dilakukan untuk melindungi kerahasiaan dan integritas sistem rekam medis elektronik, salah satunya dengan teknologi \textit{Distributed Ledger Technology} (DLT).

Salah satu sistem rekam medis elektronik yang sudah mengimplementasikan teknologi DLT dalam sistemnya adalah MedRec yang dikembangkan oleh MIT. Sistem ini mengimplementasikan 
teknologi \textit{blockchain} dan \textit{capability based access control} (CapBAC) yang disimpan sebagai smart contract untuk mengatur pihak-pihak yang dapat mengakses rekam medis \parencite{medrec}. 
Namun implementasi \textit{blockchain} pada sistem rekam medis elektronik memberikan tantangan baru berupa tambahan \textit{latency} saat mengakses data karena menggunakan konsensus 
\textit{Proof-of-Work}, sehingga rata-rata dibutuhkan waktu hingga 10 detik untuk menetapkan menetapkan kontrak baru dan merespons.

Solusi lainnya yang diberikan adalah DecMed, implementasi sistem rekam medis elektronik berbasis IOTA yang dikembangkan oleh I Putu Bakta Hari Sudewa, mahasiswa S1 Teknik Informatika 
Institut Teknologi Bandung angkatan 2021. Penggunaan IOTA memberikan penyelesaian permasalahan sistem rekam medis elektronik berbasis \textit{blockcchain} yang memiliki komputasi mahal.
Implementasi DecMed sendiri menggunakan mekanisme akses kontrol \textit{capability based access control} (CapBAC) yang dikombinasikan dengan \textit{proxy re-encryption} dan 
\textit{revoke time}.

Implementasi DecMed juga masih menyisakan kelemahan dalam sistemnya, di antaranya manajemen pemulihan aksea yang rumit karena pasien akan diberikan 12 frasa pemulihan dan harus menyimpan
keseluruhan frasa, dan juga waktu pencabutan akses yang statis dan tidak bisa diganti oleh pasien setelah ditetapkan. Untuk mengatasi kedua masalah ini, diusulkan penggunaan 
User Managed Access (UMA) sebagai mekanisme \textit{authorization} dan juga implementasi \textit{consent portal} bagi pasien yang dapat memberikan kemudahan bagi pengguna untuk melakukan
manajemen akses rekam medisnya kepada pihak lain.