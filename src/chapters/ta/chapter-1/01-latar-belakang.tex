\section{Latar Belakang}

\sloppy

Penyelenggaraan tata kelola klinis di Indonesia saat ini mewajibkan setiap fasilitas pelayanan kesehatan (Fasyankes) untuk menyelenggarakan rekam medis elektronik (RME) secara mandiri sebagai basis dokumentasi medis pasien \autocite{permenkes2022rekam}.
Kewajiban ini mengharuskan seluruh fasyankes, mulai dari tingkat pratama hingga tingkat lanjut, untuk melakukan transisi penuh dari rekam medis berbasis kertas menjadi sistem elektronik.
Dengan demikian, arsitektur data kesehatan nasional bertumpu pada ribuan sistem RME yang dikelola oleh masing-masing entitas penyedia layanan kesehatan \autocite{kemkesBerita}.

% Penyelenggaraan tata kelola klinis di Indonesia saat ini mewajibkan setiap fasilitas pelayanan kesehatan (fasyankes) untuk menyelenggarakan rekam medis elektronik (RME) secara mandiri sebagai basis dokumentasi medis pasien \autocite{permenkes2022rekam}.

Agar data yang tersebar di berbagai sistem RME tersebut dapat dipertukarkan untuk kesinambungan pelayanan, pemerintah menetapkan bahwa setiap sistem elektronik milik fasyankes harus memiliki kemampuan interoperabilitas melalui platform SATUSEHAT \autocite{permenkes2022satusehat}.
SATUSEHAT merupakan ekosistem pertukaran data kesehatan yang menghubungkan sistem informasi atau aplikasi dari seluruh anggota ekosistem digital kesehatan Indonesia termasuk fasyankes, regulator, penjamin, dan penyedia layanan digital \autocite{kemkesSATUSEHATEkosistem}.

Secara arsitektur, SATUSEHAT didesain sebagai \textit{National Health Data Platform} yang berfungsi sebagai hub integrasi tunggal dan integrasi secara terpusat. 
Meskipun penyimpanan fisik data klinis (RME) tetap berada di masing-masing fasyankes, namun mekanisme pertukaran data, indeks pasien, dan manajemen persetujuan (\textit{consent management}) dikelola secara terpusat oleh Kementerian Kesehatan \autocite{cetakbirukemenkes}.

SATUSEHAT mengadopsi standar FHIR (\textit{Fast Healthcare Interoperability Resources}) untuk interoperabilitas data \autocite{kemkesSATUSEHATEkosistem}. 
FHIR sendiri adalah standar dan regulasi pertukaran data rekam medis yang dibuat oleh HL7 dan sudah diadopsi oleh World Health Organization (WHO) \autocite{fhirspec}.
Meskipun FHIR adalah standar internasional yang agnostik terhadap arsitektur (bisa digunakan secara sentral maupun terdistribusi), implementasi di Indonesia sangat bergantung pada server FHIR terpusat.

% Penyelenggaraan pelayanan kesehatan di suatu negara memerlukan mekanisme pencatatan pelayanan kesehatan dan informasi medis pasien yang disimpan dalam bentuk rekam medis. 
% Penyelenggaraan pelayanan kesehatan yang dinamis dan membutuhkan kolaborasi antar-fasilitas pelayanan kesehatan mendorong terciptanya sistem rekam medis berbasis digital atau biasa disebut Rekam Medis Elektronik (RME).
% Untuk memenuhi kebutuhan tersebut, Pemerintah Republik Indonesia sudah membangun sistem rekam medis elektronik terintegrasi yang dinamakan SatuSehat, dan diatur melalui Peraturan Menteri Kesehatan Nomor 24 Tahun 2022.

% Sistem rekam medis elektronik memerlukan sistem keamanan dan akses kontrol yang komprehensif karena berkaitan dengan data dan privasi pasien. 
% Berdasarkan Peraturan Menteri Kesehatan Nomor 24 Tahun 2022, isi rekam medis adalah milik pasien dan harus dijaga kerahasiaannya \autocite{permenkes2022rekam}. 
% Berbagai penelitian dan pengembangan sudah dilakukan untuk melindungi kerahasiaan dan integritas sistem rekam medis elektronik, salah satunya dengan teknologi \textit{Distributed Ledger Technology} (DLT).

% Salah satu sistem rekam medis elektronik yang sudah mengimplementasikan teknologi DLT dalam sistemnya adalah MedRec yang dikembangkan oleh MIT.
% Sistem ini mengimplementasikan teknologi \textit{blockchain} dan \textit{capability based access control} (CapBAC) yang disimpan sebagai smart contract untuk mengatur pihak-pihak yang dapat mengakses rekam medis \autocite{medrec}. 
% Namun implementasi \textit{blockchain} pada sistem rekam medis elektronik memberikan tantangan baru berupa tambahan \textit{latency} saat mengakses data karena menggunakan konsensus 
% \textit{Proof-of-Work}, sehingga rata-rata dibutuhkan waktu hingga 10 detik untuk menetapkan menetapkan kontrak baru dan merespons.

Solusi lainnya yang diberikan adalah DecMed, implementasi sistem rekam medis elektronik berbasis IOTA yang dikembangkan oleh I Putu Bakta Hari Sudewa, mahasiswa S1 Teknik Informatika 
Institut Teknologi Bandung angkatan 2021. Penggunaan IOTA memberikan penyelesaian permasalahan sistem rekam medis elektronik berbasis \textit{blockcchain} yang memiliki komputasi mahal.
Implementasi DecMed sendiri menggunakan mekanisme akses kontrol \textit{capability based access control} (CapBAC) yang dikombinasikan dengan \textit{proxy re-encryption} dan  \textit{revoke time}.

Implementasi DecMed juga masih menyisakan kelemahan dalam sistemnya, di antaranya,
\begin{itemize}
    \item Pasien harus melakukan \textit{scan} suatu QR code yang diberikan oleh fasyankes untuk memberikan izin akses terhadap datanya di fasyankes. 
    Hal ini membuat pasien harus hadir secara langsung untuk memberikan akses.
    % \item Sistem ini hanya mendefinisikan akses kontrol untuk aktor yang telah ditentukan sebelumnya (tenaga medis, tenaga administrasi, dll). 
    % Menambahkan jenis pihak pemohon baru (misalnya, perusahaan asuransi, lembaga penelitian, layanan darurat) memerlukan modifikasi \textit{smart contract} dan aplikasi klien.
    \item Tidak ada cara bagi klien untuk mengetahui sumber daya apa yang dilindungi atau izin apa yang dibutuhkan. 
    Klien hanya dapat memberikan akses untuk data administratif dan data medisnya saja.
    
\end{itemize}
Untuk mengatasi kedua masalah ini, diusulkan penggunaan User Managed Access (UMA) sebagai mekanisme otorisasi dan juga implementasi \textit{consent portal} bagi pasien.
UMA memberikan kemampuan kepada pasien untuk memberikan izin secara asinkronus, mengelola data mana yang akan diatur aksesnya dan mengatur pihak yang akan mengakses data miliknya dan seluruh kapabilitasnya.

% jelasin consent portal
% hapus hukum yang general, lebih jelaskan yang terkait kepemilikan data di Indonesia punya rekam medis nya masing2 di fasyankes
%jadiiin paragraf untuk kelemahan
%mapping masalah ke latar belakang
% di rumusan masalah tambahkan untuk apa
% tambahin flow alur ke UMA mengapa bisa menyelesaikan Decmed  (bridging)
% atau di bab 3 jelasin mapping UMA ke decmed, di bab awal. jelasin UMA bagusnya buat apa, lalu       mapping kenapa UMA bisa menyelesaikan masalah di DecMed


% Kondisi rekam medis di Indonesia (SATUSEHAT)
% yang saat ini sudah mau menyelesaikan baru Kak Hari, tapi yang kak hari masih kurang dari segi ...
% nanti diselesaikan dengan UMA

% ganti judul dengan DLT
% cari paper yang sudah bisa menyelesaikan masalah ... dengan UMA, nanti dikaitkan dengan yang Kak Hari
% emangnya bisa langsung implementasi UMA? jadi jelaskan juga tantangan dan analisis apa aja yang perlu dilakuin sebelum implementasi, ga ujug2 cuman implementasi
% jelasin penyesuaian framework UMA ke sistem ini juga

% pakai decentralized untuk autorisasi dan jelasin tradeoff nya. tapi menjaga degree tetep biar ga terpusat

% semua istilah kaya SLA approval, auditability, dsb sebutin briefly di pendahuluan

% tujuan dan ukuran keberhasilan harus sejumlah dengan rumusan masalah
% tujuan utama dan khusus
% kalau mau nambahin yang pendaftaran fasyankes ke sistem terpusat, maka masukin juga di bab 1 bab 2

% jelasin kasus spesifiknya atau di bagian awal jelasinnya mekanisme otorisasi
