\section{Tujuan dan Ukuran Keberhasilan}

Tujuan dari Tugas Akhir ini adalah sebagai berikut:
\begin{enumerate}
\item Merancang dan mengimplementasikan \textit{authorization framework} pada sistem rekam medis elektronik terdesentralisasi berbasis IOTA (DecMed) untuk meningkatkan keamanan, privasi, dan kontrol akses data pasien.
\item Mengembangkan dan mengintegrasikan portal consent yang memungkinkan pasien mengelola persetujuan akses data secara granular, asinkronus, dan terinformasi sesuai kebutuhan dan regulasi perlindungan data pribadi di Indonesia.
% \item Mengevaluasi dan membandingkan efektivitas framework otorisasi UMA dengan mekanisme akses kontrol sebelumnya (misal: CapBAC) pada aspek usability, SLA approval, auditability, revoke time, dan fleksibilitas pemberian akses.
% \item Menganalisis dampak penerapan framework otorisasi UMA dan portal consent terhadap mitigasi risiko single point of failure, serangan siber, serta pemenuhan regulasi perlindungan data pribadi di Indonesia.
\end{enumerate}

% Ukuran keberhasilan:
% \begin{itemize}
% \item Sistem dapat mengelola pemberian dan pencabutan akses data pasien secara granular dan asinkronus.
% \item Portal consent dapat digunakan pasien untuk mengatur, memberikan, dan mencabut persetujuan akses data dengan mudah.
% \item Hasil evaluasi menunjukkan framework UMA lebih unggul atau setara dalam aspek usability, SLA approval, auditability, dan revoke time dibandingkan mekanisme sebelumnya.
% \item Sistem memenuhi persyaratan keamanan, privasi, dan regulasi perlindungan data pribadi sesuai Permenkes No. 24 Tahun 2022 dan UU PDP.
% \end{itemize}

% Tujuan yang akan dicapai pada tugas akhir ini adalah mengintegrasikan sistem rekam medis elektronik berbasis IOTA dengan \textit{consent portal} bagi pasien berbasis kerangka otorisai User Managed Access (UMA) 2.0. 
% Implementasi ini diharapkan mampu meningkatkan keamanan dari segi kerahasiaan dan integritas rekam medis elektronik, serta memberikan pemahaman terkait perbandingan akses kontrol UMA dengan akses kontrol lainnya dari sisi \textit{usability}, \textit{SLA approval}, \textit{auditability}, dan \textit{revoke time}.