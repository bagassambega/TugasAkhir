
\section{Metodologi}

Metodologi yang digunakan dalam Tugas Akhir ini terdiri dari beberapa tahapan utama yang dipecah menjadi sub-tahapan deskriptif sebagai berikut:

% Pada tahapan ini, dilakukan eksplorasi dan pengumpulan informasi terkait perkembangan sistem rekam medis elektronik yang sudah
%     ada dan mengidentifikasi kelemahannya. Permasalahan-permasalahan yang ditemukan kemudian diidentifikasi dan diolah
%     menjadi kebutuhan-kebutuhan yang perlu diselesaikan.
    
%     \item \textbf{Perancangan Arsitektur dan Desain Sistem} \newline
%     Berdasarkan kebutuhan-kebutuhan yang sudah didefinisikan, dilakukan analisis dan perancangan sistem yang dapat
%     menjawab kebutuhan-kebutuhan tersebut.
    
%     \item \textbf{Pengembangan Prototipe Sistem} \newline
%     Sistem yang sudah dirancang pada tahap sebelulnya diimplementasikan dengan menggunakan teknologi yang sesuai dengan 
%     kebutuhan dan hasil rancangan.
    
%     \item \textbf{Pengujian dan Evaluasi} \newline
%     Hasil implementasi prototipe sistem diuji untuk memastikan kesesuaian tujuan dengan implementasi dan seluruh kebutuhan
%     sudah berhasil terpenuhi.

% \end{enumerate}

% kurang spesifik, bikin poin2 untuk menjelaskan langkah detail


\begin{enumerate}
    \item \textbf{Studi Literatur dan Analisis Sistem Saat Ini}
    \begin{enumerate}
        \item \textbf{Analisis Sistem DecMed}: Pada tahapan ini, dilakukan eksplorasi dan pengumpulan informasi terkait DecMed, termasuk arsitektur, mekanisme akses kontrol, alur pemberian dan pencabutan akses, dan \textit{tech-stack} yang digunakan oleh sistem DecMed saat ini.
        \item \textbf{Analisis Authorization Framework}: Melakukan studi literatur dan analisis mengenai \textit{authorization framework} secara umum, termasuk prinsip kerja, komponen utama (autentikasi, otorisasi, \textit{consent management}), serta penerapannya di sistem rekam medis elektronik dan sistem terdistribusi lainnya.
    \end{enumerate}

    \item \textbf{Perancangan Arsitektur dan Desain Sistem}
    \begin{enumerate}
        \item \textbf{Perumusan Spesifikasi Kebutuhan}: Menyusun spesifikasi kebutuhan sistem berdasarkan hasil identifikasi masalah dan tujuan penelitian.
        \item \textbf{Desain Arsitektur Sistem}: Mendesain arsitektur sistem rekam medis elektronik terdesentralisasi yang mengintegrasikan \textit{authorization framework} dan \textit{consent portal}.
        \item \textbf{Pembuatan Model Alur Otorisasi dan Consent}: Menyusun model alur otorisasi, pemberian dan pencabutan akses, serta pengelolaan persetujuan pasien secara granular dan asinkronus.
    \end{enumerate}

    \item \textbf{Pengembangan Prototipe Sistem}
    \begin{enumerate}
        \item \textbf{Implementasi Prototipe}: Mengimplementasikan prototipe sistem sesuai desain yang telah dirumuskan menggunakan teknologi yang sesuai.
        \item \textbf{Integrasi Komponen}: Melakukan integrasi antara komponen \textit{authorization framework} dan \textit{consent portal} dengan sistem DecMed.
        \item \textbf{Penyusunan Skenario Pengujian}: Menyusun skenario penggunaan dan pengujian untuk memastikan seluruh fitur berjalan sesuai kebutuhan.
    \end{enumerate}

    \item \textbf{Pengujian dan Evaluasi}
    \begin{enumerate}
        \item \textbf{Pengujian Fungsionalitas}: Melakukan pengujian fungsionalitas sistem, termasuk pemberian dan pencabutan akses, pengelolaan persetujuan, dan keamanan data.
        \item \textbf{Analisis Hasil Evaluasi}: Menganalisis hasil pengujian dan evaluasi untuk menilai pencapaian tujuan dan ukuran keberhasilan yang telah ditetapkan.
    \end{enumerate}
\end{enumerate}