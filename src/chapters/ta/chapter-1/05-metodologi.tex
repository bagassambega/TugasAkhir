\section{Metodologi}

Metodologi yang digunakan dalam Tugas Akhir ini:

\begin{enumerate}
    \item \textbf{Studi Literatur dan Kondisi Saat Ini} \newline
    Pada tahapan ini, dilakukan eksplorasi dan pengumpulan informasi terkait perkembangan sistem rekam medis elektronik yang sudah
    ada dan mengidentifikasi kelemahannya. Permasalahan-permasalahan yang ditemukan kemudian diidentifikasi dan diolah
    menjadi kebutuhan-kebutuhan yang perlu diselesaikan.
    
    \item \textbf{Perancangan Arsitektur dan Desain Sistem} \newline
    Berdasarkan kebutuhan-kebutuhan yang sudah didefinisikan, dilakukan analisis dan perancangan sistem yang dapat
    menjawab kebutuhan-kebutuhan tersebut.
    
    \item \textbf{Pengembangan Prototipe Sistem} \newline
    Sistem yang sudah dirancang pada tahap sebelulnya diimplementasikan dengan menggunakan teknologi yang sesuai dengan 
    kebutuhan dan hasil rancangan.
    
    \item \textbf{Pengujian dan Evaluasi} \newline
    Hasil implementasi prototipe sistem diuji untuk memastikan kesesuaian tujuan dengan implementasi dan seluruh kebutuhan
    sudah berhasil terpenuhi.

\end{enumerate}